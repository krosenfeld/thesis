\chapter{Conclusion} \label{chap:conclusion}

One might argue that the space of discovery in Astronomy is spanned by resolution and 
wavelength.  Technology has enabled us to contemplate phenomena at wavelengths far outside 
what the realm of our human senses, revolutionizing fields like high energy astrophysics,
cosmology, and the study of the inter stellar-medium.  Resolution then reveals the hidden
details and helps us to interpret what these new eyes see.  A recent example is 
the application of interferometry at sub-mm wavelengths with ground breaking telescopes like the SMA, 
the amazing new power of ALMA, and the EHT revolution.  
While the science motivating chapters \ref{chap:v4046_structure} 
through \ref{chap:radial_flows} may be different than in the final chapter of this thesis, 
I am left wondering and waiting for what will we see when we look only a little bit closer.

Emboldened by this spirit, we shall conclude not only with summaries of the work presented here,
but also some brief discussion of recent developments in the field.  In chapter \ref{chap:v4046_structure} we 
used the SMA to study the protoplanetary disk 
around the close binary V4046 Sgr.  This disk featured a narrow ring of large mm-sized grains, 
but an extended gas rich disk.  Detailed modeling of these features demonstrated that they were 
phenomenologically consistent with the viscous spreading of the gas and the concentration through pressure 
drag forces of the dust.  This disk is a single example among a number, and more such examples continue to be 
found \citep{isella07,panic09,andrews12,degregorio-monsalvo13,ke14} although it is not universal 
\citep{huelamo15}.  A somewhat related phenomenon are the 
severely asymmetric dust  ``horse-shoes'' that have been recently observed at multiple wavelengths
\citep{vandermarel13,fukagawa13,isella13,casassus15,marino15}.  Thought to be perhaps generated by gigantic 
3D Rossby wave instabilities \citep{regaly13}, these are another striking example from the interplay between
the gas and dust in protoplanetary disks.  Another exciting example are the narrow gaps in the dust continuum
of HL Tau. This striking image was obtained using ALMA baselines as long as 15\,km and possible causes include
planet-disk interactions and condensation fronts \citep{brogan14,zhang15}.

In chapter \ref{chap:hd163296} we explored the vertical structure of a protoplanetary disk using 
observations that showed clear, resolved signs of the vertical temperature gradient in the disk.
The search for azimuthal temperature variations are a natural extension \citep{isella13} and 
some evidence found \citep{vanderplas14}.  There is much activity to continue designing
observations that can probe the ambient conditions 
for planet formation beyond temperature from the ionization environment \citep{cleeves14,teague15} to 
turbulence \citep{hughes11,guilloteau12,flock15,simon15}.

To conclude our work on protoplanary disks, we discussed in \ref{chap:radial_flows} observational indicators of 
radial gas flows within the holes of transition disks.  \citet{casassus15} have studied the gas kinematics for
the curious case of HD142527 using the CO $J = 6-5$ line and find that their data is consistent with
accretion reaching close to free-fall velocities.  The gas surface density of transition disks have also 
been recently studied using observations taking at sub-millimeter wavelengths 
\citep{bruderer14,zhang14,canovas15,perez15,vandermarel15} and will help determine how these structures are formed
\citep{bruderer13}.  Molecular disk winds, outflows, and binary systems are all recently observed 
with interesting gas kinematics \citep{klaasen13,dutrey14,williams14,salyk14,czekala15}.  Looking to the 
future, well resolved, highly sensitive observations of spectral lines may constrain the strength of MRI-drive
turbulence \citep{simon15}, query the active sites of planet formation \citep{kleeves15,ober15}, or reveal
circumplanetary disks \citep{perez15}.  The morphology of spectral line emission can provide be a powerful 
tool for studying astrophysics beyond kinematics.

The Event Horizon Telescope continues to break ground and reports exciting new science with every
campaign from the early detections of AGN at mm wavelengths \citep{doeleman05} to more recently structures
at the event horizon of a supermassive black hole \citep{doeleman08,doeleman09,doeleman13}.
The 2015 campaign represented a new scale of operation, incorporating new telescopes and
technologies.  In chapter \ref{chap:aphids} we reported on the status of the APHIDS software which
will post-process the phased-SMA data and make the sampling rate compatible with the rest of the EHT
network.  This problem is not limited to the SMA and we hope that the development done with APHIDS will
prove useful for other VLBI experiments. 

This thesis is a small expression of 
the science that resolution at these wavelengths can provide and only an exploration for the 
hard and promising work ahead.  
