\chapter{Conclusion} \label{chap:conclusion}

We conclude with brief summaries of the preceding chapters along with some discussion of recent developments 
in the field.

In Chapter \ref{chap:v4046sgr_structure} we 
used the SMA to study the protoplanetary disk 
around the close binary V4046 Sgr.  This disk featured a narrow ring of large mm-sized grains along with
an extended gas rich disk.  Detailed modeling of these features demonstrated that they were 
consistent with the viscous spreading of the gas and the concentration of the dust via pressure 
drag forces.  This disk is one published example among a few and more similar systems continue to be
identified \citep{isella07,panic09,andrews12,degregorio-monsalvo13,zhang14}.  A related phenomenon are the 
severely asymmetric dust  ``horse-shoe'' disks that have been observed at multiple wavelengths
\citep{vandermarel13,fukagawa13,isella13,casassus15,marino15}.  Thought to be perhaps generated by gigantic 
3D Rossby wave instabilities \citep{regaly12}, these are another striking example of the interplay between
the gas and dust in protoplanetary disks.  Another particular exciting example is HL Tau where recent
observations have revealed narrow gaps in the dust 
continuum (see Figure \ref{fig:hltau}. This image was obtained using ALMA baselines as long as 15\,km 
and possible causes include
planet-disk interactions and condensation fronts \citep{brogan15,zhang15}.

\begin{figure}
\centering
\includegraphics[width=0.5\linewidth]{chapters/HLTau_nrao.jpg}
\caption{This ALMA observation of the HL Tau disk revealed multiple rings and gaps \citep{brogan15}.}
\label{fig:hltau}
\end{figure}

In Chapter \ref{chap:hd163296} we explored the vertical structure of a protoplanetary disk using 
observations that exhibited clear signs of the kinetic temperature changing vertically in the disk.
The search for azimuthal temperature variations are a natural extension of this work \citep{isella13} and 
some evidence has been found \citep{vanderplas14}.  There is much activity designing 
observations to probe the ambient conditions 
for planet formation beyond temperature from the local ionization environment \citep{cleeves14,teague15} to 
turbulence \citep{hughes11,guilloteau12,flock15,simon15}.

To conclude our work on protoplanary disks, we discussed in Chapter \ref{chap:radial_flows} observational 
indicators of 
radial gas flows within the holes of transition disks.  \citet{casassus15} have studied the gas kinematics for
the curious case of HD142527 using the CO $J$ = $6$-$5$ line and found that their data is consistent with
accretion reaching close to free-fall velocities.  The gas surface density of transition disks have also 
been recently studied using observations taking at sub-millimeter wavelengths 
\citep{bruderer14,zhang14,canovas15,perez15,vandermarel15} and will help determine how these structures are formed
\citep{bruderer13}.  Molecular disk winds, outflows, and binary systems are all recently observed 
with interesting gas kinematics \citep{klaassen13,dutrey14,williams14,salyk14,czekala15}.  Looking to the 
future, well resolved, highly sensitive observations of spectral lines may constrain the strength of MRI-drive
turbulence \citep{simon15}, query the active sites of planet formation \citep{cleeves15,ober15}, or reveal
circumplanetary disks \citep{perez15}.

The ground-breaking Event Horizon Telescope continues to report exciting, new science with every
campaign from the early detections of AGN at mm wavelengths \citep{doeleman05} to structures
at the event horizon of a supermassive black hole \citep{doeleman08,doeleman12}.
The 2015 campaign represented a new scale of operations, incorporating more telescopes and
fielding new technologies.  In Chapter \ref{chap:aphids} we reported on the status of the APHIDS software which
will post-process the phased SMA data and make the sampling rate compatible with the rest of the EHT
network.  This problem is not limited to the SMA and we hope that the development done with APHIDS will
prove useful for other sites and VLBI experiments. 
