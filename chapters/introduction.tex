\chapter{Introduction}\label{chapter:introduction}

\section{Structure of Protoplanetary Disks}

The canonical picture of star formation breaks this process into several stages.  Picture first a large 
molecular cloud within which there are complicated sub structures: clumps and cores of 
concentrated gas and dust.  Over time these self-gravitating units will contract,  the material collapses into 
a disk to conserve angular momentum, and nascent stars ignite their thermonuclear engines.  Jets, envelopes, 
winds, and outflows punctuate this 
process which may likely be disturbed by the swift pull of a passing star or the constant presense of a nearby 
neighbor.  Whole fields, techniques, and technologies have been developed to study single chapters of this 
complicated story and ours begins near the very end when only the disk and young star remain.

Over time, this disk of gas and dust will dissipate leaving trace debris: astroids and comets, dust and planets.
So while we may be at the end of the star's formation, this protoplanetary disk is the stage where planets are
born.  Their composition and structure provides the crucial initial conditions that must enable a diverse 
population of planets.  However, theories of planet formation are still developing and require the consideration 
of physical processes acting on many physical scales and over a huge range of times.  As such, there 
work to test and develop these ideas is very diverse, ranging from experiments on grain interactions in 
microgravity to space missions exploring the larger rocks that remain as testament to the formation of our 
own Solar system.  

Over the past several decades, observations from interferometers operating at millimeter wavelengths have made
significant contributions to understanding the structure of these disks.  The early maps of CO made using 3 to 5 
telescopes in an array at 
Owens Valley revealed bound, rotating disks with diameters 
similar to the dimensions of own solar system 
\citep{sargent87,weintraub87,weintraub89,sargent91,koerner93a,koerner93b,koerner95}.  The emission pattern
matching Keplerian 
rotation was then used to measure some systems' dynamical masses 
\citep{dutrey98,guilloteau98,guilloteau99,simon00} and
maps of multiple species to constrain the two-dimensional structure 
\citep{aikawa99,aikawa01,vanzadelhoff01,aikawa02}.   Soon arrays detected the giant central dust holes of 
``transition disks''
\citep{hughes07,brown09,andrews11}.

There exists several techniques for constraining the structure of protoplanetary disks.  Some geometric 
properties such as size, inclination, and orientation can be calculated from an image.  The amount of dust 
in a disk can also be estimated by assuming a dust opacity and measuring the mm-wave flux.  Alternatively, one 
may try to solve the inverse problem: assume some disk model, calculate the resulting emission, and then compare
it to what is observed.  Often the density, temperature, and velocity profiles of the disks are assumed to 
follow power laws or some other parametric function.  There also exist some grids and models that try to 
calculate abundances in a self-consistent manner.  Additionally, researchers usually either try to constrain 
populations using 
surveys \citep[i.e.][]{andrews11,harris14} or present detailed studies of single, interesting disks.  The work
presented in Chapter \ref{chap:blagh} to \ref{chap:blagh} takes the latter approach and uses simple parametric 
models to investigate the structure of a choice few, but very interesting, protoplanetary disks.

V4046 Sgr is a system with two stars closely orbiting each other.  Around this tight binary is a bright
protoplanetary disk.  In chapter \ref{chap:V4046Sgr} we present Submillimimeter Array (SMA) observations of this 
disk 
that resolve for the
first time a large central cavity.  In addition, we show that the distribution of gas and dust are non-commitant 
with each other: the gas disk is large and extended while the mm-sized dust grains are concentrated in a narrow 
ring.  Furthermore, we show that dust within ring is not completely cleared.  We argue that this configuration is 
the hallmark of old disk sculpted by at least one large, embedded planet.  HD163296 also has a very large, bright gaseous disk, but no resolved cavity.  In chapter \ref{chap:hd163296} we 
use 
science verification data from the Atacama Large Millimimeter Array (ALMA) to study the vertical structure of 
this disk.  We argue that resolved features in the images highlight how the temperature of the gas changes with
its distance from mid-plane and explore additional, nuanced aspects of the disk structure and how they may 
affect the observables.   In chapter \ref{chap:radial_flows} we consider what impact radial motions of gas within transition disk 
holes may have upon spectral line emission.  We discuss several indicators of such radial flows along 
with possible sources of confusion such as warps in the disk.


\section{Basics of Radio Interferometry}
\subsection{Visibilities}
\subsection{Channel Maps}
\subsection{VLBI}

\section{Sample Rate Conversion for the EHT}

The Event Horizon Telescope (EHT) uses VLBI to observe supermassive black holes, obtaining resolution high enough
to resolve event horizon scales of nearby targets.  These observations provide a 
unique glimpse into one of the most extreme environments in our universe.  Using telescopes in Hawaii, Arizona, 
and California, the collaboration has already detected structure at these scales for Sgr A$^\ast$ in the galactic 
center and the giant elliptical galaxy M87 \citep{doeleman08,doeleman09}.  These data have provided indirect 
evidence for existence of an event horizon \citep{broderick15} along with constraints on the degree of 
magnetization in the jet of M87 \citep{kino15} and the orientation of Sgr A$^\ast$ \citep{broderick11}.  EHT 
observations promise to provide tests of general relativity \citep{luminet79,johannsen10,bambi13,broderick14} as 
well as new studies of black hole time variability \citep{doeleman09} and black hole accretion \citep{chan15}.

The bright future of the EHT rests on its key ability to expand by adding more telescopes and
incorporating new technologies into its operations.  The March 2015 observing campaign was a major milestone in
the EHT expansion with participation from the LMT in Mexico, CARMA in California, the SMA and JCMT in Hawaii, 
SMT in Arizona and 30m at Pico Valeta in Spain.  However, the data taken at the SMA must be processed before it
can be correlated with the rest of the array.  In chapter \ref{chap:aphids} we describe the \APHIDS\, pipeline
which performs a sampling rate conversion using a cluster of Graphcs Processing Units (GPUs).  This software
is intended to be a long-term solution to enable VLBI with the SMA and has applications for other sites such 
as ALMA that operate at a sampling rate that is not compatible with the standard VLBI data product.

