\chapter{Introduction}\label{chapter:introduction}

\section{Structure of Protoplanetary Disks}

The canonical picture of star formation breaks this process into several stages.  Picture first a large 
molecular cloud within which there are complicated sub structures: clumps and cores of 
concentrated gas and dust.  Over time these self-gravitating units will contract,  the material collapses into 
a disk to conserve angular momentum, and nascent stars ignite their thermonuclear engines.  Jets, envelopes, 
winds, and outflows punctuate this 
process which may likely be disturbed by the swift pull of a passing star or the constant presense of a nearby 
neighbor.  Whole fields, techniques, and technologies have been developed to study single chapters of this 
complicated story and ours begins near the very end when only the disk and young star remain.

Over time, this disk of gas and dust will dissipate leaving trace debris: astroids and comets, dust and planets.
So while we may be at the end of the star's formation, this protoplanetary disk is the stage where planets are
born.  Their composition and structure provides the crucial initial conditions that must enable a diverse 
population of planets.  However, theories of planet formation are still developing and require the consideration 
of physical processes acting on many physical scales and over a huge range of times.  As such, there 
work to test and develop these ideas is very diverse, ranging from experiments on grain interactions in 
microgravity to space missions exploring the larger rocks that remain as testament to the formation of our 
own Solar system.  

Over the past several decades, observations from interferometers operating at millimeter wavelengths have made
significant contributions to understanding the structure of these disks.  The early maps of CO made using 3 to 5 
telescopes in an array at 
Owens Valley revealed bound, rotating disks with diameters 
similar to the dimensions of own solar system 
\citep{sargent87,weintraub87,weintraub89,sargent91,koerner93a,koerner93b,koerner95}.  The emission pattern
matching Keplerian 
rotation was then used to measure some systems' dynamical masses 
\citep{dutrey98,guilloteau98,guilloteau99,simon00} and
maps of multiple species to constrain the two-dimensional structure 
\citep{aikawa99,aikawa01,vanzadelhoff01,aikawa02}.   Soon arrays detected the giant central dust holes of 
``transition disks''
\citep{hughes07,brown09,andrews11}.

There exists several techniques for constraining the structure of protoplanetary disks.  Some geometric 
properties such as size, inclination, and orientation can be calculated from an image.  The amount of dust 
in a disk can also be estimated by assuming a dust opacity and measuring the mm-wave flux.  Alternatively, one 
may try to solve the inverse problem: assume some disk model, calculate the resulting emission, and then compare
it to what is observed.  Often the density, temperature, and velocity profiles of the disks are assumed to 
follow power laws or some other parametric function.  There also exist some grids and models that try to 
calculate abundances in a self-consistent manner.  Additionally, researchers usually either try to constrain 
populations using 
surveys \citep[i.e.][]{andrews11,harris14} or present detailed studies of single, interesting disks.  The work
presented in Chapter \ref{chap:blagh} to \ref{chap:blagh} takes the latter approach and uses simple parametric 
models to investigate the structure of a choice few, but very interesting, protoplanetary disks.

\section{The Event Horizon Telescope}

\section{Basics of Radio Interferometry}
\subsection{Visibilities}
\subsection{Channel Maps}
\subsection{VLBI}

\section{Thesis Outline}

