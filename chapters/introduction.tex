\chapter{Introduction}\label{chapter:introduction}

Thomas Young's double split experiment is often a student's first formal introduction to the ideas of 
coherence and interference.  By observing the famous fringe pattern, Young could study the wave-like nature of 
photons and aspects of our physical world that operate on quantum scales.  Today, the techniques of interferometry
are ubiquitious over a wide array of technical fields.  Even within astronomy the scientific applications are 
diverse: interferometers  search for extrasolar planets, find subtle ripples in the early universe and watch 
stars explode.  But in all of these examples the astronomer is searching for small details in our sky that, 
without the resolving power of interferometry, might remain hidden.  Going along with this theme, this thesis 
presents four studies using observations at millimeter wavelenghts that work to reveal fine structures in disks 
around young stars and massive black holes.  We cover a wide variety of techniques, from enabling the production 
of useful data products to the basic interpretation of never before seen features.

Millimeter interferometry is intimately tied to the earlier development of techniques and technologies 
used at longer wavelengths.  The essential operation combines the signals from pairs of telescopes to produce a 
measure that is sensitive to some angular scale on the sky.  This scale shrinks as the distance
between the two elements grows.  By assembling an array with multiple 
telescopes, an observer can accumulate information about many spatial scales and reconstruct an image.  However,
the information contained in this image will be limited to the scales originally measured by the interferometer.
Images with the best fidelity come from arrays that have many elements spaced at a wide variety of 
distances and the highest resolution comes from the telescopes with the farthest separation.

\begin{figure}[t!]
\plottwo{chapters/sma_small.jpg}{chapters/ALMA_NRAO_padilla_14.jpg}
\caption{{\it Left:} The SMA in its compact configuration with an antenna transporter on the right hand side 
({\it photo credit:} J. Weintroub).  {\it Right:} Part of the ALMA array along with the high site control room and correlator building ({\it photo credit}: ALMA (ESO/NAOJ/NRAO), C. Padilla).}
\label{fig:sma_alma}
\end{figure}

Each pair of antennas in an array forms a baseline vector (measured in wavelengths) with length $D_\lambda$ and 
components $(u,v,w)$.  The signal voltages, $V_1(t)$ and $V_2(t)$, measured by the individual 
elements are correlated against each other:
\begin{equation}
r(\tau) = \lim_{T \to \infty} \frac{1}{2T} \int_{-T}^T V_1(t) V_2^\ast(t-\tau) d\tau.
\end{equation}
This operation is often done on site with data streamed directly to a correlator and the resulting 
measurement, or visibility, is stored on disk.  However, for Very Long Baseline Interferometry (VLBI), the 
radio telescopes can be separated by entire continents and real-time correlation is not usually feasible.  For 
these experiments, the signal voltages must be recorded and the visibilities calculated at some later time.  The 
Event Horizon Telescope (EHT) is a leader in the developing field of millimeter VLBI and relies on existing 
observatories.  The difficulty of observing at these short wavelengths increases as the atmosphere becomes 
noiser, time variable, and more opaque.  Consequently, facilities that can operate at these wavelengths are a 
precious resource.

The Submillimeter Array \citep[SMA;][]{ho04} is one such world class array.  Commisioned in 2003, the SMA is 
located on the top of Maunakea on the Big Island of Hawaii at an altitude of 13,386 feet.
It is composed of eight 6 meter diameter dishes which, like many interferometers, can be moved into different 
configurations so that there are elements as close as 15 meters or as far as 500 meters (see Figure 
\ref{fig:sma_alma}).  Its recievers can operate at wavelengths from 430\,microns to 2\,millimeters and 
historically have provided up to 4\,GHz of bandwidth.  Chapter \ref{chap:v4046sgr_structure} features an 
exquisite dataset from the SMA that combines observations taken by multiple configurations.  A major upgrade to 
the array is increasing the bandwidth to 8\,GHz, enabling more sensitive observations and larger spectral 
coverage.  To participate in VLBI experiments, the SMA can operate as a ``phased array'', essentially turning 
the interferometer into a single dish telescope.  Chapter \ref{chap:aphids} addresses a key step of how to 
integrate the SMA into millimeter VLBI observations such as those taken by the EHT.

Located 4450\,km away in Chile is the Atacama Large Millimeter Array (ALMA).  This facility began early
science operations in 2011 and has been commissioned in a staged fashion where each observing cycle offers more 
capabilities.  Once completed, ALMA will host an impressive 66 antennas, 64 of will be 12\,meter dishes and
the remaining smaller 7\,meter ones (see Figure \ref{fig:sma_alma}).  We used publicly available data from 
this instrument in Chapters \ref{chap:hd163296} and \ref{chap:radial_flows}.  A fully extended configuration 
will include baselines up to 16\,kilometer and the array will observe at wavelengths from 400\,microns to 
7\,millimeters.  The ALMA Phasing Project (APP) will enable this giant array to join VLBI experiments and 
fringes to APEX were found during a January 2015 test \citep{matthews15}.

\section{Structure of Protoplanetary Disks}

Both the SMA and ALMA are excellent observatories with which to study young stars.  Within a few hundred parsecs 
of the Sun there are multiple star forming regions and stellar moving groups.  Their relative proximity to us 
makes it possible to resolve physical structures in these regions and our great observatories have returned 
spectacular glimpses of the complicated processes that produce a star.  The canonical picture of star formation 
breaks this process into several stages \citep{shu87,mckee07}.  Picture first a large 
molecular cloud made up of complex sub-structures: clumps and cores of 
concentrated gas and dust.  Over time these self-gravitating units will contract,  the material collapses into 
a disk to conserve angular momentum, and budding stars ignite their thermonuclear engines.  Jets, winds, and 
outflows punctuate this process.  The system might also be perturbed by the swift pull of a nearby passing star 
or the constant presence of stellar partners forming a hierarchical system.  Whole fields, techniques, and 
technologies have been developed to study single chapters of this 
complicated story and ours begins near its end when only the disk and young star remain.

Over time, this disk of gas and dust will dissipate leaving trace debris: astroids and comets, dust and planets.
So while we may be at the end of the star's formation, this protoplanetary disk is the stage where planets are
born.  Protoplanetary disk composition and structure provides the crucial initial conditions that must 
enable a diverse 
population of planets.  Theories of planet formation are still developing and require the consideration 
of physical processes acting on many physical scales and over a huge range of times.  As such, the
work to test and develop these ideas is quite varied, ranging from theoretical calculations to controlled 
experiments in the lab to the frontiers of space exploration.

\begin{figure}[t]
\vspace{4cm}
\includegraphics[width=0.5\textwidth]{chapters/koerner93.jpg}
\caption{\citet{koerner93} imaged the disk around GM Aur using the Owens Valley Millimeter array. This figure
from their paper 
show maps of the $^{13}$CO $J$=2$-$1 emission which on the left is integrated over velocity (or frequency) and 
the right has been weighted by the channel velocity.}
\end{figure}

Over the past several decades, observations from interferometers operating at millimeter wavelengths have made
significant contributions to understanding the structure of these disks.  The early maps of the CO emission
made using 3 to 5 telescopes in an array at Owens Valley revealed bound, rotating disks with diameters 
similar to the dimensions of own solar system \citep{sargent87,weintraub89,koerner93}.  Emission from other 
molecules were then surveyed \citep{dutrey97} and studies of multiple isotopologues were used to constrain the 
two-dimensional structure of some disks \citep{dartois03,pietu07,qi11} aided by the development of 
chemical models \citep{aikawa99,aikawa01}.  Images of the dust also revealed interesting structures such as
giant holes centered around the star for a class of ``transition disks'' \citep{hughes07,brown09,andrews11} and 
evidence for binary interactions \citep{akeson98}.

%% figure: disk structure cartoon?

The main two ingredients in the disk are gas and dust.  The majority of the disk mass will be in molecular 
hydrogen, but observations focus on easier to detect trace molecules such as $^{12}$CO and its isotopologues.
The dust grains are detected via their bulk thermal emission which, when the emisison is optically thin,  
can be used to measure of dust mass \citep{hildebrand83}.  Interferometric observations taken at 
millimeter wavelengths can resolve structure of nearby protoplanetary disks on scales of a few tens to 
hundreds of astronomical units.  The bulk motion of material in the disk can also be 
studied using these same telescopes by studying how the pattern of line emission
from molecules such as CO changes with frequency \citep{beckwith93}.  The velocity of the gas relative to the 
observer changes as a function of position in the disk. To an observer scanning over frequency, different parts 
of the disk seem to light up when its Doppler shifted emission matches the observer's frequency window.

There exists several techniques for constraining the structure of protoplanetary disks.  Some geometric 
properties such as size, inclination, and orientation can be calculated from an image.  The amount of dust 
in a disk can also be estimated by assuming a dust opacity and measuring the millimeter wave flux.  
Alternatively, one may try to solve the inverse problem: assume some disk model, calculate the resulting 
emission, and then compare the model's prediction to what is observed \citep[i.e.][]{dartois03,qi03}.  Often the 
density, temperature, and velocity profiles of the disks are assumed to 
follow power laws or some other parametric function.  There also exist some model grids that provide more
physically motivated and self consistent disk structures \citep{dalessio05,woitke10}.  Additionally, researchers 
usually either try to constrain populations using 
surveys \citep[i.e.][]{andrews11,oberg11,harris12} or present detailed studies of single, 
interesting disks.  The work
presented in chapters \ref{chap:v4046sgr_structure} to \ref{chap:radial_flows} takes the latter approach and 
uses simple parametric 
models to investigate the structure of a choice few, but very interesting, protoplanetary disks.

V4046 Sgr is a system with two stars closely orbiting each other and around this tight binary is a bright
protoplanetary disk.  In Chapter \ref{chap:v4046sgr_structure} we present SMA observations 
of this disk that reveal a large central cavity in the dust.  In addition, we show that the dust 
distribution is significantly different than that of the gas: the gas disk is large and extended while the 
millimeter sized dust grains are concentrated in a narrow ring.  Furthermore, the amount of dust left inside
the cavity is inconsistent with photoevaporative scenarios.  We argue 
that this configuration is the hallmark of an old disk that is being sculpted by at least one unobserved 
companion. 

HD163296 also has a very large, gas rich disk, but no central cavity has yet been observed.  In Chapter 
\ref{chap:hd163296} 
we use science verification data from ALMA to study the vertical structure of 
this disk.  We argue that certain resolved features in the images can be attributed to how the temperature of 
the gas changes with
its distance from the disk mid-plane and explore how additional, nuanced aspects of the disk structure may affect
the data.  Chapter \ref{chap:radial_flows} then considers what impact radial motions of gas within transition 
disk holes may have upon spectral line emission.  We discuss several indicators of such radial flows along 
with possible sources of confusion such as warps in the disk.  One disk has provided intriguing
evidence of such flows \citep{casassus13,casassus15} and future observations may be able to detect others.

\begin{figure}[t]
\plotone{chapters/EHT-baselines.png}
\caption{A selection of EHT baselines viewed looking back from the Galactic Center where Sgr$^\ast$ lies.  The 
2015 campaign included sites 
from Hawaii, California, Arizona, Mexico, and Spain. ({\it image credit}: the EHT collaboration)}
\label{fig:eht_baselines}
\end{figure}

\section{Sample Rate Conversion for the EHT}

The Event Horizon Telescope (EHT) observes supermassive black holes with baselines long enough
to resolve event horizon scales of nearby targets.  These VLBI observations provide a 
unique glimpse into one of the most extreme environments in our universe.  Using telescopes in Hawaii, Arizona, 
and California, the collaboration has already detected structure at these scales for Sgr A$^\ast$ in the Galactic 
Center and the black hole at the center of the giant elliptical galaxy M87 
\citep{doeleman08,doeleman09}.  These data provide indirect 
evidence for the existence of an event horizon \citep{broderick15} along with constraints on the degree of 
magnetization in the jet of M87 \citep{kino15} and the orientation of Sgr A$^\ast$ \citep{broderick11}.  EHT 
observations promise to provide tests of general relativity \citep{luminet79,johannsen10,bambi13,broderick14} as 
well as new studies of black hole time variability \citep{doeleman09} and accretion \citep{chan15}.

The bright future of the EHT is partialy driven by its ability to add more telescopes and
incorporate new technologies.  The recent March 2015 observing campaign marked a major milestone in
the EHT expansion with participation from the Large Millimeter Telescope (LMT) in Mexico, Combined Array for 
Research in Millimeter-wave Astronomy (CARMA) in California, the SMA and James Clerk Maxwell Telescope (JCMT) in
Hawaii, SMT in Arizona and the 30m dish at Pico Valeta in Spain.  Figure \ref{fig:eht_baselines} shows the
impressive geographic span of the EHT including sites in Chile and at the South Pole that will soon be 
incorporated into the array.  However, the data taken at the SMA must be 
processed before it
can be correlated with the rest of the array.  In Chapter \ref{chap:aphids} we describe the APHIDS pipeline
which performs a sampling rate conversion using a cluster of Graphcs Processing Units (GPUs).  This software
is intended to be a long-term solution enabling VLBI with the SMA.
