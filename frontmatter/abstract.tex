% the abstract

Interferometric observations at millimeter wavelengths provide a precious, detailed view 
of certain astrophysical objects.  This thesis is composed of studies that both rely on and enable this 
technique to study the structure of planet-forming disks and soon image the closest regions around 
super-massive black holes.  Young stars form out of a cloud of gas and dust that, before its eventual dissipation,
flattens to a disk.  However the disk population is diverse and recent high-resolution images have 
revealed a wide variety of interesting features.  To understand these observations we use detailed radiative 
transfer models to motivate various physical scenarios.  First we identify a set of traits in the disk around
V4046 Sgr that marks the coupled progression of the gas and dust distributions in the presence of at least one
embedded companion.  Next, we investigate how the vertical temperature structure of a disk can be spatially 
resolved and apply our framework to observations of the disk around HD163296.  Lastly, we show how large-scale
radial flows of gas may be observable and question how this phenomenon might be distinguished from other 
scenarios such as warps or outflows.  The last chapter summarizes the APHIDS project which changes the sampling 
rate of data taken at the SMA so that it may be used for VLBI campaigns.
